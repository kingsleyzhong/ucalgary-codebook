\chapter{Various}

\section{Intervals}
	\kactlimport{IntervalContainer.h}
	\kactlimport{IntervalCover.h}
	\kactlimport{ConstantIntervals.h}

\section{Misc. algorithms}
	\kactlimport{MergeSort.h}
	\kactlimport{BinarySearch.h}
	\kactlimport{TernarySearch.h}
	\kactlimport{LIS.h}
	\kactlimport{FastKnapsack.h}

\section{Dynamic programming}
	\kactlimport{KnuthDP.h}
	\kactlimport{DivideAndConquerDP.h}

\section{Debugging tricks}
	\begin{itemize}
		\item \verb@signal(SIGSEGV, [](int) { _Exit(0); });@ converts segfaults into Wrong Answers.
			Similarly one can catch SIGABRT (assertion failures) and SIGFPE (zero divisions).
			\verb@_GLIBCXX_DEBUG@ failures generate SIGABRT (or SIGSEGV on gcc 5.4.0 apparently).
		\item \verb@feenableexcept(29);@ kills the program on NaNs (\texttt 1), 0-divs (\texttt 4), infinities (\texttt 8) and denormals (\texttt{16}).
	\end{itemize}



\section{Optimization tricks}
	\verb@__builtin_ia32_ldmxcsr(40896);@ disables denormals (which make floats 20x slower near their minimum value).
	\subsection{GCC Builtins}
	For these builtins, you can append \texttt{l} or \texttt{ll} to the function names to get the \texttt{long} or \texttt{long long} version.
		\begin{itemize}
			\item \lstinline{int __builtin_ffs(int x)} Returns one plus the index of the least significant 1-bit of $x$. Returns 0 if $x=0$.
			\item \lstinline{int __builtin_clz(unsigned int x)} Returns the number of leading 0-bits in $x\neq0$, starting at the most significant bit position. $\left\lfloor\log_2\left(n\right)\right\rfloor=$ \texttt{31 - \_\_builtin\_clz(n)}  
			\item \lstinline{int __builtin_ctz(unsigned int x)}  Returns the number of trailing 0-bits in $x\neq0$, starting at the most significant position.
			\item \lstinline{int __builtin_clrsb(int x)} Returns the number of leading redundant sign bits in $x$, i.e. the number of bits following the most significant bit that are identical to it. There are no special cases for 0 or other values.
			\item \lstinline{int __builtin_popcount(unsigned int x)} Returns the number of 1-bits in $x$. (Slow on x86 without SSE4 flag)  
			\item \lstinline{int __builtin_parity(unsigned int x)} Returns the parity of $x$, i.e. the number of 1-bits in $x$ modulo 2. 
			\item \lstinline{uintN_t __builtin_bswapN(uintN_t x)} Returns $x$ with the order of the bytes reversed. $N=16,32,64$      
		\end{itemize}
	\subsection{Bit hacks}
		\begin{itemize}
			\item \verb@x & -x@ is the least bit in \texttt{x}.
			\item \verb@for (int x = m; x; ) { --x &= m; ... }@ loops over all subset masks of \texttt{m} (except \texttt{m} itself).
			\item \verb@c = x&-x, r = x+c; (((r^x) >> 2)/c) | r@ is the next number after \texttt{x} with the same number of bits set.
			\item \verb@rep(b,0,K) rep(i,0,(1 << K))@ \\ \verb@  if (i & 1 << b) D[i] += D[i^(1 << b)];@ computes all sums of subsets.
			\item \verb@(x & (x-1)) == 0@ Check if x is a power of 2. Note: 0 is edge case.
		\end{itemize}
	\subsection{Pragmas}
		\begin{itemize}
			\item \lstinline{#pragma GCC optimize ("Ofast")} will make GCC auto-vectorize loops and optimizes floating points better.
			\item \lstinline{#pragma GCC target ("avx2")} can double performance of vectorized code, but causes crashes on old machines.
			\item \lstinline{#pragma GCC optimize ("trapv")} kills the program on integer overflows (but is really slow).
		\end{itemize}
	\kactlimport{int128.h}
	\kactlimport{FastMod.h}
	\kactlimport{FastInput.h}
	\kactlimport{BumpAllocator.h}
	\kactlimport{SmallPtr.h}
	% \kactlimport{BumpAllocatorSTL.h}
	% \kactlimport{Unrolling.h}
	% \kactlimport{SIMD.h}
